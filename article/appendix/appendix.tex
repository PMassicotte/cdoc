\documentclass[12pt,a4paper]{scrartcl}
\usepackage[utf8]{inputenc}
\usepackage{amsmath}
\usepackage{amsfonts}
\usepackage{amssymb}
\usepackage{graphicx}

\usepackage[bottom = 1in, left = 0.5in, right = 0.5in, top = 1in]{geometry}

\usepackage[english]{babel}
\usepackage[autostyle]{csquotes}
\usepackage{mathptmx}

\usepackage[labelfont=bf]{caption}

\usepackage[default, scale=0.95]{opensans} %% Alternatively
%% use the option 'defaultsans' instead of 'default' to replace the
%% sans serif font only.
\usepackage[T1]{fontenc}

\addto\captionsenglish{\renewcommand{\figurename}{Supplementary Fig.}}
\addto\captionsenglish{\renewcommand{\tablename}{Supplementary Table}}

\title{My neat title here}
\subtitle{Supplementary materials}
\date{}

\begin{document}
\maketitle

\begin{figure}[h]
	\centering
	\includegraphics[scale = 1]{../../graphs/appendix1}
	\caption{Barplot showing the number of unique observations for (\textbf{A}) principal regions and (\textbf{B}) ecosystems.}
\end{figure}

\clearpage
\newpage

% latex table generated in R 3.3.1 by xtable 1.8-2 package
% Wed Aug 17 11:46:18 2016
\begin{table}[ht]
\centering
\begin{tabular}{ccccr}
  \hline
Wavelength (nm) & Intercept & Slope & $R^2$ & $n$ \\ 
  \hline
253 & -1.27 & 0.28 & 0.9874 & 30 \\ 
  254 & -1.26 & 0.28 & 0.9876 & 5102 \\ 
  280 & -0.98 & 0.37 & 0.9915 & 132 \\ 
  300 & -0.54 & 0.49 & 0.9955 & 239 \\ 
  320 & -0.26 & 0.64 & 0.9980 & 134 \\ 
  325 & -0.19 & 0.69 & 0.9987 & 336 \\ 
  330 & -0.14 & 0.74 & 0.9992 & 27 \\ 
  340 & -0.08 & 0.86 & 0.9997 & 29 \\ 
  355 & 0.02 & 1.08 & 0.9999 & 1183 \\ 
  365 & 0.11 & 1.27 & 0.9990 & 45 \\ 
  375 & 0.14 & 1.50 & 0.9978 & 239 \\ 
  380 & 0.17 & 1.63 & 0.9968 & 899 \\ 
  400 & 0.28 & 2.25 & 0.9903 & 308 \\ 
  412 & 0.36 & 2.69 & 0.9841 & 1037 \\ 
  420 & 0.44 & 3.02 & 0.9786 & 59 \\ 
  440 & 0.64 & 3.95 & 0.9608 & 219 \\ 
  443 & 0.66 & 4.10 & 0.9566 & 946 \\ 
   \hline
\end{tabular}
\caption{Coefficients of the linear regressions between absorption 
coefficents at 350 nm and other wavelengths. Each regression includes a total 
of 2321 observations. All regression have p-value < 0.00001.  $n$ represents 
the number of observations that were reported at this wavelength.} 
\end{table}


\clearpage
\newpage

\begin{figure}[h]
	\centering
	\includegraphics[scale = 1]{../../graphs/appendix2}
	\caption{Heat map showing the determination coefficients ($R^2$) of the linear regressions between absorption values for each pair of wavelengths between 250 and 500 nm ($n = 63001, 0.83 \le R^2 \le 1$). Each regression is based on 2194 observations. Note that the diagonal of the plot shows $R^2 = 1$. }
\end{figure}

\clearpage
\newpage

\begin{figure}[h]
	\centering
	\includegraphics[scale = 1]{../../graphs/appendix3}

	\caption{Log-linear relationship bewteen $a_{CDOM}(350)$ and DOC divided by ecosystems. The blue line is the fitted values of a linear model $y = log(x)$. The shaded areas represent the 95\% confidence intervals. A total of 11562 observations are distributed across all panels (light gray dots in the background show the same global relation presented in Fig. 4).}
\end{figure}

\clearpage
\newpage

\begin{figure}[h]
	\centering
	\includegraphics[scale = 1]{../../graphs/appendix4}
	\caption{World map showing the 359 km buffer zone identified by the segmentation analysis (Fig. 5).}
\end{figure}

\clearpage
\newpage

\begin{figure}[h]
	\centering
	\includegraphics[scale = 1]{../../graphs/appendix5}
	\caption{Barplots showing how the data is distributed temporally across North and South hemisphere ($n = 12707$).}
\end{figure}

\end{document}
