%!TEX root = massicotte_et_al.tex

\section*{Discussion}
\label{sec:Discussion}

\subsection*{Relationships between DOC and $a_{\text{cdom}}$ along the aquatic continuum}

DOC concentration and $a_{\text{cdom}}$ measurements are often used as proxies to characterize the quantity and the quality of the DOM pool in aquatic ecosystems. The coefficient of correlation between $a_{\text{cdom}}$ and DOC concentration was highly dependent on the wavelength used (Fig. 2). Similar, inverse relationship between correlation coefficient and absorption wavelength has been shown previously in freshwaters and estuaries \citep{Asmala2012, Peacock2014}. This decoupling of CDOM and DOC at higher wavelengths is likely due to the higher photobleaching of DOM at wavelengths above 400 nm \citep{Osburn2009}. The observed relationship decreases more gradually in freshwater and coastal waters, whereas in ocean the DOC-CDOM coupling weakens considerably already above around 350 nm. This supports the hypothesis of the role of photobleaching because the residence time is longer and light attenuation lower in the ocean compared to other systems, increasing the photobleaching potential \citep{DelVecchio2002, Swan2012, Nelson2013}. The weak relationship between DOC concentration and CDOM absorption at wavelengths above 350 nm is relevant for remote sensing applications, which typically operate at wavelengths above 400 nm. Typical solution to overcome the inherently poor DOC-CDOM relationship is to use multiple wavelengths (satellite bands) and additional variables in the algorithms \citep{DelCastillo2008, Mannino2008}.

In freshwater ecosystems, absorption properties of the DOM pool were found to be highly correlated with DOC concentration. A likely explanation for this is that inland water ecosystems are known to be highly connected with their surrounding terrestrial catchment \citep{Wiens2002, Frenette2012} which contribute to the delivery of large quantities  of highly colored DOM \citep{Massicotte2011EA, Lambert2016, Cole2007}. In contrast, weaker relationships between absorption properties and DOC were observed in coastal and oceanic ecosystems (Fig. 5). \citet{Stedmon2015b} suggested that kinetic processes (mixing, photochemical oxidation and microbial degradation) can operate at different rates on DOC and CDOM fractions of the DOM pool which in result can cause a decoupling and nonlinear relationships between these two pools. At a global scale, \acdom{350} correlated strongly to the carbon content of the DOM pool (Fig. 4, supplementary Fig. 3). However, the relationships between DOC and \acdom{350} became gradually decoupled as the DOM pool was transported toward the oceans. One possible explanation is that as residence times of the water systems increase from headwater streams to oceans,  also increases the importance of photochemical processes resulting in preferential removal of the colored fractions of the DOM pool \citep{Vahatalo2004, Moran2000, Bittar2015, Hansen2016, Weyhenmeyer2012}. This hypothesis can be further supported by exploring the negative relationship between DOC and $a_{\text{cdom}}$ documented by \citet{Nelson2013} (Supplementary section 7). One characteristic of these sample points is that they are mainly located in open oceans where DOC concentrations are low (< 100 \mmol).  Hence, one possible explanation behind this negative relationship is that carbon content and color are decoupled due to long exposure to solar irradiation and thus high potential of photobleaching. Below the surface the bleaching decreases rapidly, but production of DOC remains relatively high and the overall positive relationship is reestablished.

In wetlands, rivers and estuaries, a large fraction of the DOM pool originates from soil erosion, surface runoff \citep{Massicotte2011EA, Lambert2015a} as well as subsurface inputs. A large proportion of terrestrially-derived DOM is known to be of high molecular weight \citep{Heinz2015}. The so-called size-reactivity continuum concept proposed by \citet{Amon1996} suggests that once exported from soils to aquatic environments these highly reactive macromolecules are prone to intense biological and physical transformations. For example, photodegradation and bacterial degradation have been shown as active processes altering the color of DOM in natural aquatic ecosystems \citep{Moran2000, Helms2014}. Moreover, it was found that these processes had a greater influence upstream within rivers where the DOM pool is mainly composed of soil-derived carbon \citep{Weyhenmeyer2012, Seidel2015}. Hence, the DOM pool in lakes and oceans is a mixture of both old refractory and fresh reactive organic material. The changing balance between these two major fractions of the DOM pool can be seen in the high variability in SUVA along the aquatic continuum with a general decrease in absorption, concentration and absorptivity. As shown by numerous earlier studies, both DOC concentration and CDOM absorption decrease when moving along the aquatic continuum (Fig. 3). Wetland ecosystems exhibited between two and three orders of magnitude higher values in both DOC and \acdom{350} compared to oceans. Interestingly, the change in specific absorbance (\suva{350}) is more gradual (Fig. 3C), from average value of around 1 in wetlands to around 0.1 \suvagram in oceans. The loss of specific absorbance when moving downstream along the aquatic continuum suggests continuous bleaching of terrestrial DOM, along with the distance from the terrestrial source. Such decoupling between DOC and CDOM were also observed in the open ocean \citep{Nelson2002}. When predicting \acdom{350} from DOC concentration, it is apparent that modeled CDOM is underestimated at the higher end of the observed values and overestimated at the lower end (Supplementary. Fig. 4A). This discrepancy in CDOM-DOC relationship between higher and lower end of the DOC concentration gradient is apparent also in the residuals of the log-linear fit (Supplementary Fig. 4B). Ecosystems with lower DOC concentrations (such as oceans and coastal systems) have more negative residuals, whereas ecosystems with high DOC concentrations (wetlands, rivers) have more positive residuals. This trend also supports previous findings that along the aquatic continuum, photodegradation processes decrease CDOM absorption relatively more than DOC concentration \citep{Vahatalo2004} because light absorbing compounds, or absorbing parts of larger molecules, are selectively degraded, while there is continuous production of new DOC from aquatic production that dilutes and eventually replaces the original terrestrial DOM dominance \citep{Markager2011}.

\subsection*{DOM characteristics in the transition from headwaters to oceans}

During its transport from headwaters to oceans, DOM pool is in constant change due to microbial respiration and production \citep{DelGiorgio1997, Kritzberg2006a, Berggren2010}, sedimentation and flocculation \citep{Sholkovitz1976,VonWachenfeldt2008}, production by photosynthetic organisms \citep{Descy2002, Kritzberg2005, Lapierre2009} and UV photodegradation \citep{Benner1999, Amado2006, Zhang2009}. As these processes operate simultaneously, they drive the fate of DOM transiting in the various habitats of aquatic ecosystems \citep{Sondergaard2004, Massicotte2013LOFE}. Based on the size-reactivity continuum model proposed by \citet{Amon1996}, it was expected that these degradation processes would actively act to decrease the molecular weight of organic matter and subsequently its reactivity along the aquatic continuum. In agreement with this conceptual model, we find in inland waters high \suva{254} values (Fig. 5), suggesting relatively high proportion of aromatic compounds in the DOM pool \citep{Weishaar2003}. This is likely due to elevated lateral connectivity with surrounding terrestrial landscape and organic matter inputs from the tributaries \citep{Massicotte2011EA, Lambert2016}. Our results further agree with findings from \citet{Lapierre2013} who found a positive relationship between \acdom{440} and the concentrations of biologically and photochemically degradable DOC in boreal aquatic ecosystems. This is suggesting that the highest rates of photochemical and biological processing are occurring when the DOM pool is dominated by macro molecules originating from the landscape which have a high absorbance at visible wavelengths. The decrease in SUVA, as terrestrially-derived DOM is transiting in the mosaic of aquatic ecosystems, infers that high molecular weight molecules are degraded into smaller and more refractory molecules. Interestingly, we found that the observed decrease in reactivity is also occurring in coastal shelf seas (Fig. 5). After this breakpoint, it is likely that freshly produced material from phytoplankton become the dominant fraction of the DOM pool which is known to be characterized by low molecular weight and colorless molecules. Even though the optical signal of the autochthonous DOM might be similar to terrestrial DOM \citep{Yamashita2004}, the magnitude of the production of autochthonous DOM is similar to degradation processes, resulting in only minor apparent changes in the optical characteristics of the oceanic DOM pool (Fig. 5).

\subsection*{Processing of CDOM across the salinity gradient}

A rich literature uses salinity as a tracer to estimate the conservative behavior of optical properties of DOM (e.g. \cite{Kowalczuk2010, Asmala2016}). However, current findings are based on the assumption that dilution is the main process responsible for the decrease in DOM properties. Contrary to most studies, our results show that DOM dynamics deviates from the expected conservative behavior (Fig. 6) suggesting that concurrent degradation and production processes continually operate in conjunction with conservative mixing \citep{Markager2011, Goncalves2015}. To emphasize the deviations of \suva{254} from conservative mixing along the salinity gradient, we fitted a piecewise regression which revealed breakpoints at salinities around 9 and 27. In agreement with finding from \citet{Goncalves2015}, our results evidence that there are at least two distinct phases of processing at low and high salinity. 

Between salinity 0 and 8.7 we observed a rapid decrease in \suva{254} that may be partially explained by salt-induced DOM flocculation in the confluence of freshwater and seawater \citep{Sholkovitz1976, Sondergaard2003}. Flocculation occurs at low salinities in estuaries and its efficiency reflects the characteristics of the riverine organic matter (e.g. iron content or DOM molecular weight; \cite{Mayer1982, Forsgren1996, Asmala2014a}). However, it is unlikely that flocculation is the sole process causing deviations from conservative mixing. Increasing residence times of water bodies across the continuum from river to the sea will also increase the importance of processes such as photo- and bacterial degradation which are known to act selectively on loss of CDOM prior to discharge into the sea \citep{Weyhenmeyer2012}. Hence, a more plausible scenario is that the observed kinetic of CDOM upon intrusion in marine ecosystems is a result of the combined effect of flocculation, photodegradation and biodegradation processes (see conceptual plots in Fig. 6). This salinity range covers productive estuarine and coastal waters where the selective degradation of non-colored DOM, or higher production of CDOM compared to DOC likely drives this trend. For example organisms from all trophic levels are known to contribute with CDOM \citep{Steinberg2004, Stedmon2014} although microbially mediated transformation of the freshly produced DOM is likely dominating the net increase of CDOM \citep{Rochelle-Newall2002, Yamashita2004}.

\subsection*{Major differences between CDOM absorption spectra from freshwater and seawater}

\citet{Loiselle2009} developed a method to identify the wavelength regions of CDOM spectra with the largest changes in the spectral slope (parameter $S$ in equation \ref{eq:cdom_model}). We found that averaged freshwater and marine spectral slope curves were different from each other (Fig. 7). Both curves show strongly increasing slope values at low wavelengths (260-295 nm, segments I and IV). Only ocean slope values decreased rapidly right after increasing between 292-350 nm (segment V). The large drop in slope values in freshwater occurs after 365 nm (segment III). These spectral regions are the most dynamic ones, indicating that the largest deviations from the general spectral slope occur at these wavelengths. This information could be used to update our current knowledge of optical indices inferred from the spectral data. We suggest using spectral slopes between 260-295 and 365-475 nm for freshwater systems, and 260-295 and 295-350 nm for marine systems. Because these wavelengths ranges present consistent changes in slope values across a wide variety of different locations, it is likely that they could be used as proxies for e.g. distinction between similar DOM samples, or quantifying biogeochemical processing of DOM.

\subsection*{Limitations and future research}

Our study uses data from an extensive literature survey to better understand the distribution of DOM along the aquatic continuum at global scale. All ecological studies tackling problems at global scales need to assume a certain level of abstraction that is necessary to discover general trends which comes at the cost of losing site specific details. For example, it is likely that water temperature, watershed characteristics, stream order and many other environmental variables play key roles in the dynamics of DOM. However, for the present global analysis it is impossible to gather all this specific information from the published datasets, as there are too much variability in the auxiliary data coverage and in the extent of metadata provided in the publications. Despite this, conclusions can be drawn based on the analysis of the spatial and temporal distribution of the data extracted from this study (Supplementary Fig. 5). Albeit our effort to extract all available information from the literature and open repositories, a striking finding is that aquatic ecosystems in the southern hemisphere (n = 855, 7\%) are highly under represented compared to those in northern hemisphere (n = 11 953, 93\%). Also, no data prior to 2010 was available from southern hemisphere. Furthermore, continental Africa (n = 603, 5\%), Asia (n = 423, 3\%) and South America (n = 0, 0\%) are poorly represented in the dataset, all of which contain significant inland waters and are of major significance in global carbon cycle. It is evident that the relevant data is unevenly distributed across the globe and as a result, our understanding about the biogeochemical cycling of DOM on the global scale is inevitably limited. Another important fact is that the majority of samples are taken during summer (Supplementary Fig. 5). Because this corresponds to  the productive season with maximum primary production, this might lead to bias towards autochthonous signal. Collecting the dataset for this study demonstrated that a great deal of scientific data is still not made openly available. We therefore encourage the scientific community to continue with the efforts of making the research data available in open access databases so that the data collected can continue to contribute to progress in the field in the best possible way. Furthermore, from the analysis of spectral CDOM absorption and its relationship with DOC concentration, it is evident that the level of detail acquired from single wavelength measurements is considerably inferior compared to the use of spectral information. As the modern computational capabilities do not present obstacles for utilizing spectral analyses, we strongly recommend researchers to use and develop methods that use the full potential contained in the CDOM spectra, such as the spectral slope curve \citep{Loiselle2009} and the Gaussian decomposition \citep{Massicotte2016MC}. Our data shows breakpoints in the DOM optical properties on the salinity gradient, but fully mechanistic understanding about the underlying factors is unclear. There is need for dedicated studies unraveling the role of the few key processes affecting the DOM transformation along the aquatic continuum, most importantly about the role of salt-induced flocculation and phytoplankton-derived DOM.

\renewcommand{\abstractname}{Acknowledgements}
\begin{abstract}
This work was supported by the IMAGE (09-067259) from the Danish Council for Strategic Research (S.M.). C.S. was funded by the Danish Research Council for Independent Research (DFF-1323-00336) P. M. was supported by a postdoctoral fellowship from The Natural Sciences and Engineering Research Council of Canada (NSERC). E.A. and C.S. were supported by the BONUS COCOA project (grant agreement 2112932-1), funded jointly by the EU and Danish Research Council. We acknowledge Ciarán Murray for helpful comments on the manuscript and Claudine Ouellet who greatly helped with data collection.
\end{abstract}
