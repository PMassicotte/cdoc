%!TEX root = massicotte_et_al.tex

\section*{Material and methods}
\label{sec:Material and methods}

\subsection*{Literature survey and spatial coverage}

Web of Science, Google Scholar as well as public data repositories were searched using terms "dom", "cdom", "doc", "dissolved organic carbon", "dissolved organic matter" and "absorption" for datasets presenting original (i.e. not summarized) values of DOC and optical properties of CDOM. The minimum variables required to be included in the dataset were DOC concentration, absorbance or absorption of CDOM (or an integrative value which could be used to infer the required values, such as SUVA), geographical coordinates and time of the sampling. When not explicitly provided, geographical coordinates were estimated by hand using available sampling map in each study. For CDOM data, wavelengths and cuvette size used for the absorption measurements were also required. Using these criteria, we compiled 65 datasets containing 12 808 unique observations of simultaneous DOC concentration and absorption properties of DOM measured between 1991 and 2015 (Fig. 1, Table 1). A total of 4 712 observations with complete CDOM spectra (i.e. continuous measurements along a range of wavelengths) and 8 096 observations with CDOM absorption measurement at discrete wavelengths were extracted.

Observations were distributed on four different continents and in seven oceans (Fig. 1, Supplementary Fig. 1A). A large proportion of the data was located in river and ocean ecosystems and to a lesser extent in estuaries, wetlands and lakes regions (Supplementary Fig. 1B). Oceanic observations were spread over all major basins. In North America, dense clusters of observations were mostly located in large rivers and estuaries of the East Coast of United States. and along the Gulf of Mexico from the Rio Grande, Texas to Anclote Island, Florida (Fig. 1). To the north, observations were mainly located in Alaska along the Mackenzie and the Tanana rivers. Additional observations were found for the St. Lawrence river, the Great Lakes and the Hudson Bay. In Europe, a large portion of the samples came from the Baltic Sea and the North Sea as well as from lakes in Sweden. Few observations were also located around Greenland. In Russia, the data originated from the river observatory stations on the Ob, Yenisei, Lena and Kolyma rivers as well as around the Laptev and Siberian shelf. In Asia, observations were located in South Korea rivers, in lake Taihu in China and on west coast of Taiwan. In Africa, most of the observations were located in Congo, Niger, Zambezi and the Ogooué rivers. In Australia, data stemmed from the St. Vincent gulf and the border of Timor Sea.

\subsection*{Ecosystem classification}

Each observation was assigned to a defined ecosystem using either the sampling location or the salinity when available (Supplementary Fig. 1B). Observations with salinity values were classified as follows: river (salinity $\leq$ 0.5), estuary (0.5 < salinity $\leq$ 5), coastal (5 < salinity $\leq$ 30), ocean (salinity > 30). Based on available information, observations were classified as follows (Supplementary Fig. 1A): river ($n$ = 4 896), ocean ($n$ = 3 439), coastal ($n$ = 1 405), estuary ($n$ = 1 367), wetlands ($n$ = 954), lake ($n$ = 747). Note that 81\% of the oceanic samples were located at depth shallower than 1 800 m.

\subsection*{Data processing and metrics calculation}

Absorbance by CDOM were converted to absorption coefficients and expressed per meter using equation \ref{eq:cdom} \citep{Kirk1994}:

\begin{equation}
  a_{\text{CDOM}}(\lambda) = \frac{2.303 \times A(\lambda)}{L}
  \label{eq:cdom}
\end{equation}

where \acdom{\lambda} is the absorption coefficient (m$^{-1}$) at wavelength $\lambda$, A($\lambda$) the absorbance at wavelength $\lambda$ and $L$ the pathlength of the optical cell in meters. Given that UV-visible absorption spectra of CDOM decrease approximately exponentially with increasing wavelength, a simple exponential model (equation \ref{eq:cdom_model}) has been used to extract quantitative information about optical properties of CDOM \citep{Jerlov1968, Bricaud1981, Stedmon2001}:

\begin{equation}
  a_{\text{CDOM}}(\lambda) = a_{\text{CDOM}}(\lambda_0)e^{-S(\lambda - \lambda_0)} + K
  \label{eq:cdom_model}
\end{equation}

where $a_{\text{CDOM}}$ is the absorption coefficient (m$^{-1}$), $\lambda$ is the wavelength (nm), $\lambda_0$ is a reference wavelength (nm) and $K$ is a background constant (m$^{-1}$) accounting for scatter in the cuvette and drift of the instrument. $S$ is the spectral slope (nm$^{-1}$) that describes the approximate exponential rate of decrease in absorption with increasing wavelengths. Specific UV absorbance (\suva{254}, \suvagram) was calculated by dividing absorbance at 254 nm by DOC concentration \citep{Weishaar2003}. The \suva{254} metric is commonly used as a proxy for assessing both chemical \citep{Weishaar2003, Westerhoff2004} and biological reactivity \citep{Berggren2009, Asmala2013} of the DOM pool in natural aquatic ecosystems. In figures, along with the \suva{254} parameter, we also present the value of specific absorption coefficient of DOC (a*, $m^2 \times mol~C^{-1}$) which is calculated by dividing the absorption coefficient measured at 254 nm by the DOC concentration expressed in mol of C per cubic meter. The motivation arises from the fact that this parameter is commonly used to parametrize bio-optical models which can be of interest for the community. For simplicity only SUVA values are discussed in the text. However, note that the conversion between SUVA and a* at a given wavelength can be done using a conversion factor of 27.64 (a* = 27.64 $\times$ SUVA). Spectral slope curves based on CDOM absorption spectra were calculated as described in \citep{Loiselle2009}. Briefly, spectral slopes were calculated over a sliding window of 21 nm along the complete spectral range using equation 2. Each calculated slope value was associated to the middle wavelength of the current sliding window.

Given the wide range of wavelengths used in each study, absorption spectra were filtered to keep measurements between 250 and 600 nm at 1 nm increment. For the~\cite{Nelson2013} dataset (Table 1), absorption spectra were only available between 275 and 600 nm and therefore not included in the spectral analysis. Five criteria were used to control the quality of absorption spectra: (1) \suva{254} had to be smaller or equal to 6 \suvagram, (2) the spectral slope ($S$, equation 2) had to be smaller than 0.08 nm$^{-1}$, (3) the determination coefficient of the fit of the spectral slope (\rr, equation 2) needed to be at least 0.95, (4) the value of \acdom{440} needed to be positive and (5) the value of \acdom{350} needed to be > 0.01 m$^{-1}$. Based on these criteria, a total of 128 absorption spectra were discarded from further analyses.

\subsection*{Estimation of \acdom{350}}

In the extracted data, we found that a wide range of different wavelengths (between 250 and 490 nm) were used as reference wavelength to report absorption coefficients of CDOM (Supplementary Table 1). To make absorption coefficients comparable among studies, an interpolation procedure was used to estimate \acdom{350} independently of the wavelength reported in each study (Supplementary section 2). This choice was motivated by the fact that absorption at 350 nm was among the most reported wavelength in the available data. An additional motivation was its central position in the wide range of wavelengths reported. To achieve the interpolation, we used an empirical approach where complete absorption spectra ($n$ = 2 431) were used to predict the value of \acdom{350} from observations measured at other wavelengths (Supplementary Fig. 2i). This was done by regressing all the absorption values at a specific wavelength against that measured at 350 nm (ex. \acdom{254} vs. \acdom{350}). Then, the slope (Supplementary Fig. 2iB) and the intercept (Supplementary Fig. 2iC) of the linear regression were used to predict \acdom{350} from \acdom{\lambda}. Based on reported wavelengths, a total of 24 linear models were made (Supplementary Table 1). A minimum value of 0.98 for the determination coefficient (\rr) was used as a threshold to discard observations that were too far from the targeted wavelength of 350 nm. Regression coefficients used to estimate \acdom{350} in this study are presented in Supplementary Table 1, but the complete list of coefficients are also provided as a supplementary comma-separated values (CSV) file that enable the calculation of a given wavelength from another in the range of 250–500 nm. Detailed results of the procedure can be found in Supplementary section 2.

\subsection*{Estimation of distance from shoreline as a proxy for the extent of DOM processing}

We used the distance from the nearest coastline and \suva{254} to estimate DOM biogeochemical processing history ("bleaching") and reactivity for freshwater and marine samples along the aquatic continuum. Here, distance can be seen as a crude proxy for stream order or stream size, which both typically increase from inland towards river mouth at the coastline. More detailed standardized metadata was not consistently available from the published studies. The distance to the closest coastline was calculated using ocean shapefiles (resolution = 1:110000000) available on the Natural Earth website (http://www.naturalearthdata.com/) and the rworldmap R package \citep{South2011}. For inland samples (rivers) the measured distances have been assigned as a positive value whereas for marine samples, the measured distances were assigned as a negative value. Lakes were not included in this analysis, since their connectivity to larger-scale aquatic continuum is less obvious than that of rivers. Because precise geographical coordinates were not always available and thus often estimated from available maps, calculated distances have been pooled to 150 km bins which was found to distribute the observations roughly equally in each bin.

\subsection*{Statistical analysis}

Segmentation analysis were performed using the segmented R package \citep{Muggeo2003, Muggeo2008} to determine breakpoints in relationships that were presenting bi-linear patterns. CDOM metrics were calculated using the cdom R package \citep{Massicotte2016MC}. Spatial analyses were done with the rgeos R package \citep{Bivand2016}. All statistical analysis were performed in R 3.3.2 \citep{RCoreTeam2016}.
