%!TEX root = massicotte_et_al.tex

\section*{Results}
\label{sec:Results}

\subsection*{Estimation of DOC from CDOM absorption coefficients}

Estimation of DOC concentration using CDOM absorption measurements is a commonly used technique in many ecological, biogeochemical and remote sensing based studies. However, the relationship between these two key parameters is rarely evaluated at global scale. The robustness of the DOC predictions from absorption measured at different wavelengths is presented in Fig. 2. For this exercise, ecosystems were categorized into three groups as follows: freshwater (lakes, rivers, wetlands), coastal (coasts, estuaries) and ocean. This choice was motivated by the similarities in the CDOM-DOC relationships across the systems (Supplementary Fig. 3). For these three ecosystem classes, prediction of DOC as a function of \acdom{\lambda} was found to decrease monotonically with increasing wavelengths but with varying magnitude (Fig. 2). In freshwater and coastal ecosystems, the goodness of the predictions remained relatively high across the complete spectral range. In freshwater, the robustness of the relationship between \acdom{\lambda} and DOC remained relatively high and stable between 250 and 400 nm with and the coefficient of determination averaged 0.91 before decreasing to 0.69 at 500 nm. Prediction of DOC was also relatively high for coastal samples where \rr varied between 0.82 and 0.64. For ocean samples, the prediction of DOC from absorption measurements was much lower and \rr decreased rapidly from 0.63 at 250 nm to 0.02 at 500 nm.

\subsection*{Distribution of \acdom{350}, DOC and \suva{350} along the aquatic continuum}

The distributions of the main variables used to characterize the DOM pool along the freshwater-marine continuum are presented in Fig. 3. DOC concentrations ranged from 19 to 44 600 \mmol, with a median value of 769 $\pm$ 1 855 \mmol (0.05 and 0.95 quantile values: 41-2867). Absorption coefficients at 350 nm varied by three orders of magnitude between wetland and ocean ecosystems (Fig. 3A). In wetlands, median \acdom{350} was 87.2 m$^{-1}$ and decreased linearly along the freshwater-marine continuum to reach 0.08 m$^{-1}$ in ocean ecosystems. DOC concentration showed a similar negative trend along the aquatic continuum with median value decreasing from 3 250 \mmol in the wetlands to reach 50 \mmol in oceans (Fig. 3B). Median value of \suva{350} varied between 1.34 \suvagram in the wetlands and 0.09 \suvagram in oceans (Fig. 3C). Whereas \acdom{350} and DOC values both negatively decreased along the freshwater-marine gradient, \suva{350} showed similar values among inland water ecosystems (wetland, pond, lake and river) with a median value of 0.86 \suvagram. For marine-like ecosystems (coastal,, estuary and ocean) median \suva{350} was 0.15 \suvagram (Fig. 3C).

\subsection*{Global relationship between DOC and \acdom{350}}

A strong positive log-linear relationship was found between \acdom{350} and DOC (Fig. 4A, $n$ = 12 808, \rr = 0.92, p < 0.0001). At low values of DOC concentration (35 $\mu\text{mol}~L^{-1}$), predicted value of \acdom{350} was 0.03 m$^{-1}$. As DOC increased to a maximum of 44 600 $\mu\text{mol}~L^{-1}$ in wetlands, predicted \acdom{350} reached 1 097 m$^{-1}$ (Fig. 4A). The derived equation from the general log-linear model indicated that \acdom{350} increases by 9.31 m$^{-1}$ for each unit of increase in DOC concentration.

\begin{equation}
  log(a_{\text{CDOM}}(350)) = -15.09 + 9.31 \times log([DOC])
\end{equation}

The robustness of the global relationship was found to vary greatly among the different ecosystems and \rr averaged 0.68 (Fig. 4B, supplementary Fig. 3). The individual relationships between DOC and \acdom{350} for observations in ocean, coastal and lake was found to be weaker than the calculated average which caused larger scattering around the regression line at low DOC values (Fig. 4A, Supplementary Fig. 3). The weakest relationship between DOC and \acdom{350} was found in the ocean ecosystem (\rr = 0.44) whereas the strongest one was found in the wetland ecosystem (\rr = 0.94) which presented similarities with river and estuary ecosystems (Fig. 4B). An analysis of covariance (ANCOVA, using type III sum of squares) revealed that ecosystem has a significant effect on DOC, F(5, 12 799) = 401, p < 2e-16, which in this case can be interpreted as a significant difference in intercepts between the regression lines (Supplementary Fig. 3).

\subsection*{DOM characteristics along the aquatic continuum}

\suva{254} was used as an indicator of DOM chemical composition and as a proxy for both chemical and biological reactivity over 4 000 km along the aquatic continuum (Fig. 5). A piecewise regression adequately modeled the pattern observed in the data (\rr = 0.95, p < 0.0001). A significant breakpoint (p < 0.0001) was identified between 300-400 km offshore. In freshwaters and up to this coastal break point mean \suva{254} decreased rapidly from 4.79 to 1.68 \suvagram with a slope of 1.825e-03 \suvagram $\times$ km$^{-1}$ suggesting an important loss in DOM aromaticity. Beyond the identified breakpoint onward to the open ocean SUVA slowly decreased with a slope of 1.311e-06 \suvagram $\times$ km$^{-1}$. However this was not significant and \suva{254} had an average of 1.7 \suvagram. The ratio between the two identified slopes indicated that \suva{254} decreased 1 300 times more per distance in freshwater ecosystems compared to marine water ecosystems.

\subsection*{Trends in SUVA across freshwater to ocean salinity gradient }

The linear trend of \suva{254} along the salinity gradient was modeled using a piecewise regression where two different breakpoints at salinity 8.7 $\pm$ 0.3 and 26.8 $\pm$ 0.8 were identified (Fig. 6, \rr = 0.74, p < 0.0001). Between salinity 0 and 8.7, the slope of the linear regression was -0.3 indicating that \suva{254} decreased by this amount for each unit increase in salinity. Between salinity 8.7 and 26.8, \suva{254} remained stable and the slope of the regression was not significantly different from 0 (p > 0.1). Another significant slope with a value of -0.09 was found when the salinity was above 26.8 which aligned with the hypothetical conservative mixing line (see dashed line in Fig. 6).

\subsection*{Spectral differences between freshwater and marine ecosystems}

Spectral slope ($S_\lambda$) curves calculated on normalized and averaged CDOM spectra in contrasting ecosystems, which can be viewed as end-members of the global aquatic continuum; freshwater and marine systems, showed contrasting patterns (Fig. 7). $S_\lambda$ calculated from the freshwater samples had the appearance of an inverted parabola. Spectral slopes showed a linear increase in the UV-C region (segment I, < 295 nm) before reaching a plateau in the UV-B and UV-A regions (segment II, 295--365 nm). Above 365 nm, $S_\lambda$ decreased rapidly to 0.0095 nm-1 (segment III). For the marine end-member, a dominant peak in the spectral slope curve was found at approximately 280 nm (segment IV, 260--292 nm). A rapid decrease in $S_\lambda$ was observed between 292--350 nm (segment V). Beyond 350 nm, the value of $S_\lambda$ remained relatively stable (segment VI). Contrary to the freshwater curve, the quality of the computed slope, estimated using \rr, decreased from 400 nm indicating that the exponential model used to calculate $S_\lambda$ were not fully capturing the general pattern in the DOC-\acdom{\lambda} relationship (Fig. 7).
