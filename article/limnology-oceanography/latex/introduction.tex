%!TEX root = massicotte_et_al.tex

\section*{Introduction}
\label{sec:Introduction}

Physico-chemical characteristics of dissolved organic matter (DOM) drive the functioning of aquatic ecosystems at different levels. The chromophoric fraction of the DOM pool (CDOM) is a major driver of underwater light penetration \citep{Kirk1994} which modulates primary production \citep{Markager2004, Thrane2014, Seekell2015}, photochemistry and protects aquatic organisms against harmful ultraviolet (UV) radiation \citep{Hader2011}. Additionally, the carbon in the DOM pool, dissolved organic carbon (DOC), is the main source of metabolic substrates for heterotrophic bacteria and influences the composition of aquatic microbial communities \citep{Findlay2003}.

In recent decades, climate change, eutrophication and intensification of human perturbations on terrestrial systems have contributed to increased inputs of colored terrestrial DOM in aquatic ecosystems \citep{Roulet2006, Massicotte2013RSE, Weyhenmeyer2014, Haaland2010}. This has important consequences since the transformation of even a small fraction of the DOM pool can potentially have large impacts on ecosystem functioning \citep{Prairie2008}. Increases in CO$_2$ emissions \citep{Lapierre2013} and reduction in primary production due to light shading \citep{Seekell2015, Thrane2014} have already been documented as consequences of increases in terrestrial DOM at local and regional scales. However, generalizing these effects to global scales is a difficult task because our current understanding on the origin, fate and dynamics of DOM along the aquatic continuum from headwater streams to oceans is limited. Since most studies on the fate of DOM are either ecosystem or site specific, there is an indisputable need for integrative studies that will unify existing knowledge to better understand the fate and the dynamics of DOM from a broader perspective during its transport from headwaters to oceans.

DOC concentration and CDOM absorbance are now routinely measured in many ecological and biogeochemical studies. This creates an opportunity to explore the factors regulating the dynamics of DOM at global scales. In this study we performed an extensive literature survey ($n$ = 12 808) to extract datasets containing coupled DOC concentration and CDOM absorption measurements to gain insights about the spatial distribution and the compositional characteristics of the DOM pool during its transport along the aquatic continuum. We hypothesized that a strong relationship between DOC concentration and absorption properties of CDOM would be a common characteristic in freshwater ecosystems receiving large amount of colored DOM from their surrounding terrestrial catchments. We further expected that the robustness of the observed relationship would weaken as DOM is transported from freshwater to ocean ecosystems as the dominant sources will change and photochemical processes will preferentially remove CDOM over DOC \citep{Vahatalo2004}. Another key objective of this work was to gather and harmonize available published information to help the community to formulate and test new hypotheses about DOM biogeochemical cycling at global scales along the aquatic continuum.
