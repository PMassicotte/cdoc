%!TEX root = massicotte_et_al.tex

\section*{Conclusions}
\label{sec:Conclusions}

This study shows that colloidal particles (0.2-0.7 $\mu$m) can influence considerably the spectral signature of CDOM in aquatic ecosystems. DOM is a central theme in ecology and more efforts are currently used to understand its roles on the functioning of aquatic ecosystems at global scales. Although there is a growing need to combine optical datasets to formulate and test new hypotheses about DOM biogeochemical cycling, combining CDOM absorption measurements from disparate sources might not be a straightforward process because colloidal material may significantly increase scattering and subsequently the spectral signature of the CDOM pool.

\renewcommand{\abstractname}{Acknowledgements}
\begin{abstract}
	P. Massicotte was supported by a postdoctoral fellowship from The Natural Sciences and Engineering Research Council of Canada (NSERC). C. Stedmon was funded by the Danish Council for Independent Research: Natural Sciences Grant (DFF–1323-00336). S. Markager was supported by the project \textit{Dissolved organic matter - from soil to sea} funded by Aarhus University (Stiig Markager). We acknowledge Ciar\'{a}n Murray for helpful comments on the manuscript. 
\end{abstract}
