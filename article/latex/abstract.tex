\begin{abstract}
Based on an extensive literature survey containing more than 13 000 paired observations of dissolved organic carbon (DOC) concentrations and absorption of chromophoric dissolved organic matter (CDOM) distributed over five continents and seven oceans, we described the global distribution and transformation of dissolved organic matter (DOM) along the aquatic continuum from lakes to oceans. A strong log-linear relationship (R$^2$ = 0.92) between DOC and absorption of CDOM at 350 nm, \acdom{350}, was observed at global scale, but was found to be ecosystem-dependent at local and regional scales. Our results show that as the DOM pool transited toward the oceans, the robustness of the observed relation decreased rapidly (R$^2$ from 0.94 to 0.44) indicating a gradual decoupling between DOC and \acdom{350} as the connectivity between the landscape and its aquatic component decreased along the aquatic continuum. To support this hypothesis, we used the DOC-specific UV absorbance (SUVA) to characterize the reactivity of the DOM pool which decreased from  from 4.9 to 1.7 \suvagram along the aquatic continuum. Over a distance of 4000 km, a piecewise linear regression showed that the observed decrease of SUVA occurred more than 1300 times faster in freshwater ecosystems compared to marine water ecosystems, suggesting that degradation processes preferentially CDOM over DOC. The observed change in the DOM characteristics along the aquatic continuum suggests that the terrestrial DOM pool is gradually becoming less reactive, which can have profound consequences on cycling of organic carbon in aquatic ecosystems.
\end{abstract}

\smallskip
\noindent \textbf{Keywords:} Absorption, Chromophoric dissolved organic matter (CDOM), Dissolved organic carbon (DOC), Specific UV absorbance (SUVA), carbon cycling, biogeochemistry

\smallskip
\noindent \textbf{Running head:} DOM along the aquatic continuum
