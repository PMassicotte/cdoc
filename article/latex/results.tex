%!TEX root = massicotte_et_al.tex

\section*{Results}
\label{sec:Results}

\subsection*{Estimation of \acdom{350} from other wavelengths}

Between 250 and 350 nm, determination coefficients (R$^2$) gradually increased from 0.987 to 1 (Fig. 2A). After 350 nm, R$^2$ decreased rapidly to reach 0.86 at 500 nm. Between 250 and 500 nm, the regression slopes increased almost exponentially (0.28-6.99, Fig. 2B) whereas the intercepts increased more or less linearly between -1.4 and 1.5 m-1 across the spectral range (Fig. 2C). Perfect fit at 350 nm is identified using vertical dashed lines where R$^2$ = 1, slope = 1 and intercept = 0. Note that the 95\% confidence interval of the estimated slope values is hardly distinguishable which emphasises the robustness of the generated models (shaded area in Fig. 2B). Regression coefficients used to estimate \acdom{350} in this study are presented in Supplementary Table 1. Absorption coefficients measured at wavelengths higher than 412 nm were not used to estimate absorption coefficients as 350 nm because of the R$^2$ below the selected threshold of 0.98. A heat map plot showing the R$^2$ of the regressions between all possible pairs of wavelengths between 250 nm and 500 nm (1 nm increment) is presented in supplementary Fig. 2 ($n$ = 63001). Corresponding coefficients are provided as a supplementary comma-separated values (CSV) file that enable the calculation of a given wavelength from another in the range of 250-500 nm.

\subsection*{Estimation of DOC from CDOM  absorption coefficients}

Estimation of DOC concentration using CDOM absorption measurements is a commonly used technique in many ecological and remote sensing based studies. However, the relationship between these two key parameters is rarely evaluated at global scale. The robustness of the DOC predictions from absorption measured at different wavelengths is presented in Fig. 3. For this exercise, ecosystems were reclassified in as follows: freshwater (lakes, rivers, sewage, wetlands), coastal (coasts, estuaries) and ocean. For these three ecosystem classes, prediction of DOC as a function of \acdom{\lambda} was found to decrease monotonically with increasing wavelengths but with varying magnitude (Fig. 3). Due to the high humic content of the DOM pool in freshwater and coastal ecosystems, the goodness of the predictions remained relatively high along the complete spectral range. In freshwater, the robustness of the relationship between \acdom{\lambda} and DOC remained relatively high and stable between 250 and 400 nm with and averaged 0.98 before decreasing to 0.68 at 500 nm. Prediction of DOC was also relatively high for coastal samples where R$^2$ varied between 0.82 and 0.64. For ocean samples, the prediction of DOC from absorption measurements was much lower and R$^2$ decreased rapidly from 0.63 at 250 nm to 0.023 at 500 nm.

\subsection*{Distribution of \acdom{350}, DOC and \suva{350} along the aquatic continuum}

The distributions of the principal variables used to characterize the DOM pool along the freshwater-marine continuum are presented in Fig. 4. DOC concentrations ranged from 19 to 44600 \mmol, with a median value of 769 $\pm$ 1855 \mmol. Absorption coefficients at 350 nm varied by three orders of magnitude between wetland and ocean ecosystems (Fig. 4A). In wetlands, median \acdom{350} was 87.2 m$^{-1}$ and decreased linearly along the freshwater-marine continuum to reach 0.08 m$^{-1}$ in ocean ecosystems. DOC concentration showed a similar negative trend along the aquatic continuum with median value decreasing from 3250 \mmol in the wetlands to reach 50 \mmol in oceans (Fig. 4B). Median value of \suva{350} varied between 1.34 \suvagram in the wetlands and 0.09 \suvagram in oceans (Fig. 4C). Whereas \acdom{350} and DOC values both negatively decreased along the freshwater-marine gradient, \suva{350} showed similar values among inland water ecosystems (wetland, pond, lake, river and sewage) with a median value of 0.86 \suvagram. For marine-like ecosystems (coastal, brine, estuary and ocean) median \suva{350} was 0.15 \suvagram (Fig. 4C).

\subsection*{Global relationship between DOC and \acdom{350}}

A strong positive log-linear relationship was found between \acdom{350} and DOC (Fig. 5A, $n$ = 13032, R$^2$ = 0.92, p < 0.0001). At low value of DOC (35 $\mu$mol), predicted value of \acdom{350} was 0.03 m$^{-1}$. As DOC increased to a maximum of 44600 $\mu$mol in wetlands, predicted \acdom{350} reached 1097 m$^{-1}$ (Fig. 5A). The derived equation from the general log-linear model indicated that \acdom{350} increases by 9.31 m$^{-1}$ for each unit of increase in DOC.

\begin{equation}
  \log(a_{CDOM}(350)) = -15.09 + 9.31 \times \log(doc)
\end{equation}

The robustness of the global relationship was found to vary greatly among the different ecosystems and R$^2$ averaged 0.68 (Fig. 5B, supplementary Fig. 3). The individual relationships between DOC and \acdom{350} for observations in ocean, coastal, lake and sewage was found to be weaker than the calculated average which caused larger scattering around the regression line at low DOC values (Fig. 5A, supplementary Fig. 6). The weakest relationship between DOC and \acdom{350} was found in the ocean ecosystem (R$^2$ = 0.44) whereas the strongest one was found in the wetland ecosystem (R$^2$ = 0.94) which presented similarities with river and estuary ecosystems (Fig. 5B).

\subsection*{DOM reactivity along the aquatic continuum}

\suva{254} was used as a proxy for characterizing DOM chemical composition and reactivity \citep{Weishaar2003} over 4000 km along the aquatic continuum (Fig. 6). A piecewise regression adequately modelled the pattern observed in the data (R$^2$ = 0.95, p < 0.0001). A significant breakpoint (p < 0.0001) was identified at 370 $\pm$ 102 km towards the ocean from the coastline at the interface between freshwater and marine ecosystems. Between 1500 and -370 km (mostly inland waters), mean \suva{254} decreased rapidly from 4.79 to 1.68 \suvagram suggesting an important loss in DOM aromaticity. Beyond the identified breakpoint onward to the open ocean, no visible trend was observed where \suva{254} remained stable at an average of 1.7 \suvagram. The ratio between the two identified slopes indicated that \suva{254} decreased at a rate more than 1300 times faster in freshwater ecosystems compared to marine water ecosystems.

\subsection*{Trends in SUVA across freshwater to ocean salinity gradient}

The linear trend of \suva{254} along the salinity gradient was modeled using a piecewise regression where two different breakpoints at salinity 8.7 ± 0.3 and 26.8 ± 0.8 were identified (Fig. 7, R$^2$ = 0.74, p < 0.0001). Between salinity 0 and 8.7, the slope of the linear regression was -0.3 indicating that \suva{254} decreased by this amount for each unit increase in salinity. Between salinity 8.7 and 26.8, \suva{254} remained stable and the slope of the regression was not significantly different from 0 (p > 0.1). Another significant slope with a value of -0.09 was found when the salinity was above 26.8 which lined up perfectly with the hypothetical conservative mixing line (see dashed line in Fig. 7).

\subsection*{Spectral differences between freshwater and marine ecosystems}

Spectral slope (S$_\lambda$) curves calculated on normalized and averaged CDOM spectra in both end-member ecosystems showed contrasting patterns (Fig. 8). S$_\lambda$ calculated from the freshwater samples had the appearance of an inverted parabola. Spectral slopes showed a linear increase in the UV-C region (segment I, < 295 nm) before reaching a plateau in the UV-B and UV-A regions (segment II, 295 - 365 nm). Above 365 nm, S$_\lambda$ decreased rapidly to 0.0095 nm-1 (segment III) For the marine end-member, a dominant peak in the spectral slope curve was found at approximately 280 nm (segment IV, 260 - 292 nm). A rapid decrease in S$_\lambda$ was observed between 292-350 nm (segment V). Beyond 350 nm, the value of S$_\lambda$ remained relatively stable (segment VI). Contrary to the freshwater curve, the quality of the computed slope, estimated using R$^2$, decreased from 400 nm indicating that the exponential model used to calculate S$_\lambda$ were not fully capturing the general pattern in the DOC-\acdom{\lambda} relationship (Fig. 8).
