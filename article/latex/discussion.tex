%!TEX root = massicotte_et_al.tex

\section*{Discussion}
\label{sec:Discussion}

Based on an extensive literature survey that cover several characteristic global aquatic ecosystems (Fig. 1), our results shows that the spatial distribution of DOM is highly heterogeneous along the aquatic continuum. In freshwater ecosystems, absorption properties of the DOM pool were found to be highly correlated with DOC. A likely explanation for this is that inland water ecosystems are known to be highly connected with their surrounding terrestrial catchment \citep{Wiens2002, Frenette2012} which contribute to the delivery of a high quantity of highly colored DOM \citep{Massicotte2011EA, Lambert2016, Cole2007}. In contrast, weak relationships between absorption properties and DOC were observed in coastal and oceanic ecosystems (Fig. 5), suggesting that the DOM pool accumulated an intensive bleaching history operated preferentially on CDOM over DOC \citep{Vahatalo2004}.

\subsection*{Relationships between DOC and aCDOM along the aquatic continuum}

DOC and \acdom measurements are often used as proxies to characterize the quantity and the quality of the DOM pool in aquatic ecosystems. In a context of global climate change, there is an increasing interest to predict and estimate DOC and aCDOM with remote sensing techniques \citep{Kutser2005, Olmanson2016, Slonecker2016}. However, absorption coefficients of CDOM are often reported at different wavelengths (Supplementary Table 1) which make cross-studies comparisons difficult. Using an extensive dataset that includes 4802 complete profiles of absorption measurements (250-500 nm) from a wide range of ecosystems, we found out that the robustness of the relationship between DOC and \acdom{\lambda} decreased rapidly with increasing wavelengths (Fig. 3). At a global scale, \acdom{350} correlated strongly to the DOC content of the DOM pool (Fig. 4, Supplementary Fig. 3). However, the relationships between DOC and \acdom{350} became gradually decoupled as the DOM pool was transported toward the oceans. One possible explanation is that as residence times of the water systems increase from headwater streams to the ocean photochemical processes increase in importance and impart an  efficient preferential removal of  CDOM over DOC \citep{Vahatalo2004, Moran2000, Bittar2015, Hansen2016}. In wetlands, rivers and estuaries, a large fraction of the DOM pool is controlled by terrestrial inputs originating from the tributaries, soil erosion and surface runoff \citep{Massicotte2011EA, Lambert2015a}. This DOM is continually degraded by UV radiation \citep{Moran1997}. Once entering in the open oceans, a large proportion of the DOM pool has likely integrated a long history of photobleaching, causing aCDOM to be removed at a higher rate than that of DOC.

As shown by numerous studies before, both DOC concentration and CDOM absorption decrease when moving along the aquatic continuum (Fig. 4). Wetland ecosystems exhibited between two and three orders of magnitude higher values in both DOC and \acdom{350} compared to oceans. Interestingly, the change in specific absorbance (\suva{350}) is more gradual (Fig. 4C), from average value of around 1 in wetlands to around 0.1 \suvagramm in oceans. The loss of specific absorbance when moving downstream along the aquatic continuum suggests continuous bleaching of terrestrial DOM, with distance from the terrestrial source. When predicting \acdom{350} from DOC (Supplementary Fig. 5A), it is apparent that modeled CDOM is underestimated at the higher end of the observed values and overestimated at the lower end. This discrepancy in CDOM-DOC relationship between higher and lower end of the DOC concentration gradient is apparent also in the residuals of the log-linear fit (Supplementary Fig. 6B). Ecosystems with lower DOC concentrations (such as oceans and coastal systems) have more negative residuals, whereas ecosystems with high DOC concentrations (wetlands, rivers) have more positive residuals. This trend is driven by the fact that along the aquatic continuum, the rate with which CDOM is being removed is faster than that of DOC.

\subsection*{DOM reactivity during transition from headwaters to oceans}

During its transport from headwaters to oceans, microbial respiration and production \citep{DelGiorgio1997, Kritzberg2006, Berggren2010}, sedimentation and flocculation \citep{Sholkovitz1976, VonWachenfeldt2008}, production by photosynthetic organisms \citep{Descy2002, Kritzberg2005, Lapierre2009}, and UV photodegradation \citep{Benner1999, Amado2006, Zhang2009} operate simultaneously and drive the fate of DOM transiting in the various habitats of aquatic ecosystems \citep{Massicotte2013LOFE}. Based on the size-reactivity continuum model proposed by \citet{Amon1996}, it was expected that these degradation processes would actively act to decrease the molecular weight of organic matter and subsequently its reactivity along the aquatic continuum. In agreement with this conceptual model, we found out that in headwater streams, up to 1500 km toward inland continents, the DOM pool was highly aromatic (Fig. 6), presumably due to elevated lateral connectivity with surrounding terrestrial landscape and organic matter inputs from the tributaries \citep{Massicotte2011EA, Lambert2016}. Results agree with findings from \citet{Lapierre2013} who found a positive relationship between \acdom{440} and the concentrations of biologically and photochemically degradable DOC in boreal aquatic ecosystems which suggest that the highest rates of photochemical and biological processing are occurring when the DOM pool is dominated by macro molecules originating from the landscape which have a high absorbance at visible wavelengths. The decrease in SUVA as terrestrial DOM is transiting in the mosaic of aquatic ecosystems, infers that high molecular weight molecules are degraded into smaller and more refractory molecules. Interestingly, we found that the observed decrease in reactivity was also occurring in a buffer zone of 370 km around the continents once DOM entered marine water (Fig. 6, Supplementary Fig. 4). After this breakpoint, it is likely that freshly produced material from phytoplankton become the dominant fraction of the DOM pool which is known to be characterized by low molecular weight and colorless molecules. Even though the optical signal of the autochthonous DOM might be similar to terrestrial DOM \citep{Yamashita2004}, the magnitude of the production of autochthonous DOM is similar to degradation processes, resulting in only minor apparent changes in the optical characteristics of the oceanic DOM pool (Fig. 6).

\subsection*{Processing of CDOM across the salinity gradient}

A rich literature uses salinity as a tracer to estimate the conservative behavior of optical properties of DOM (see for example \cite{Kowalczuk2010} and \cite{Asmala2016}). However, current findings are based on the assumption that the DOM-pool is inert and dilution is the main process responsible for the decrease in DOM properties. Contrary to most studies (but see \cite{Goncalves2015}), our results show that DOM dynamics deviates from the expected conservative behavior (Fig. 7) suggesting that concurrent degradation and production processes continually operate in conjunction with conservative mixing. Rather than completely discarding the effect of water mixing, the fitted piecewise regression shows that there are active processes that are taking place once DOM enters marine water. For instance, \suva{254} decreased more rapidly than the hypothetical conservative mixing line (dashed line in Fig. 7). In agreement with finding from \citet{Goncalves2015}, our results evidence that there are at least two distinct phases of processing at low and high salinity.

Although still debated \citep{Markager2004, DelCastillo2000}, it was pointed out that flocculation could be an important DOM removal process \citep{Sholkovitz1976}, especially at low salinity \citep{Asmala2014a}. Between salinity 0 and 8.7 we observed a rapid decrease in SUVA254 that could be driven by flocculation which is an efficient process responsible for the removal of high molecular weight humic substance \citep{Asmala2014a}. However, it is unlikely that flocculation is the sole process causing deviations from conservative mixing. Increasing residence times of water bodies across the continuum from river to the sea will also increase the importance of processes such as photo- and bacterial degradation. Hence, a more plausible scenario is that the observed kinetic of CDOM upon intrusion in marine ecosystems is a result of the combined effect of flocculation, photodegradation and biodegradation processes (see conceptual plots in Fig. 7). At high salinity (> 26.8), the last slope of the piecewise regression lined up with the hypothetical conservative mixing line between \suva{254} and salinity. One important characteristic of deep ocean water is that the concentration of biomolecules decreases to almost unrecognizable concentrations due to a long history of degradation that took place over centuries and millennium time scale (see references in \cite{Dittmar2014}. This is suggesting that medium-aged water containing freshly and colorless DOM is being diluted once entering in contact with deep ocean water. Interestingly, the flat regression line in Fig. 7 between salinities 9 and 27 shows that there actually is net increase in \suva{254} along this segment. In the case where the net changes would be determined solely by mixing, the regression line would have a negative slope aligning with conservative mixing line with the last segment. This net increase in \suva{254} may be the result of selective degradation of non-colored DOM, or higher production of CDOM compared to DOC. There is evidence that many major groups of marine organisms are sources of CDOM \citep{Steinberg2004}. On the other hand, primary producers are not major sources of CDOM directly, but microbial transformation of the freshly produced DOM is the likely pathway leading to net increase of CDOM \citep{Rochelle-Newall2002, Yamashita2004}.

\subsection*{Major differences between CDOM absorption spectra from freshwater and seawater}

\citet{Loiselle2009} developed an approach to characterize the absorption characteristics of CDOM. This method, based on the derivative signal of the absorption signal, allows to determine the wavelength intervals where there are changes in the spectral slope (parameter S in equation 2). We found that averaged freshwater and marine spectral slope curves were different from each other (Fig. 8). Both curves show strongly increasing slope values at low wavelengths (260-295 nm, segments I and IV). Only ocean slope values decreased rapidly right after increasing between 292-350 nm (segment V). The large drop in slope values in freshwater occurs after 365 nm (segment III). These spectral regions are the most dynamic ones, indicating that the largest deviations from the general spectral slope occur at these wavelengths. This information could be used to update our current knowledge of optical indices inferred from the spectral data. We suggest using spectral slopes between 260-295 and 365-475 nm for freshwater systems, and 260-295 and 295-350 nm for marine systems. Because these wavelengths ranges present significant changes in slope values across a wide variety of different locations, it is likely that they could be used as proxies for e.g. distinction between similar DOM samples, or quantifying biogeochemical processing of DOM.

\subsection*{Limitations and Future research}

Different conclusions can be drawn based on the analysis of the spatial and temporal distribution of the data extracted from this study (Supplementary Fig. 5). Despite our effort to process all available information from the literature and open repositories, a first striking evidence is that southern hemisphere aquatic ecosystems (n = 1047, 8\%) are highly under represented compared to those in north hemisphere (n = 11985, 92\%) and no data prior to 2006 was available. Furthermore, continental Africa (n = 603, 5\%), Asia (n = 423, 3\%) and South America (n = 0, 0\%) are poorly represented in the dataset, all of which contain significant inland waters and are of major significance in global carbon cycle. This evidences that datasets are widely scattered with limited accessibility which is a major obstacle preventing to formulate and test new hypotheses about DOM biogeochemical cycling at global scales. Another important fact is that the majority of samples are taken during summer (Supplementary Fig. 5). Because this correspond the productive season with maximum primary production, this might lead to bias towards autochthonous signal. Collecting the dataset used in this study has shown that there are shortcomings in the CDOM community in making the scientific data openly available. While we fully acknowledge the considerable amount of work and funds used to acquire the data, it is vital to emphasise the importance of making spectral data available in open access databases so that the data collected can continue to contribute to progress in the field. One of the first steps to make the data available (after a reasonable period of exclusive use) would be to use the existing data portals such as Pangaea (https://www.pangaea.de/) and DRYAD (http://datadryad.org/) for uploading and storing the data. Furthermore, from the analysis of spectral CDOM absorption and its relationship with DOC, it is evident that the level of detail acquired from single wavelength measurements is considerably inferior to the use of spectral information. As the modern computational capabilities are not likely to present obstacles for utilizing spectral analyses, we strongly recommend researchers to use and develop methods that use the full potential contained in the CDOM spectra such as the spectral slope curve \citep{Loiselle2009} and the Gaussian decomposition \citep{Massicotte2016MC}. Our data shows breakpoints in the DOM optical properties on the salinity gradient, but fully mechanistic understanding about the underlying factors is unclear. There is need for dedicated studies unraveling the role of the few key processes affecting the DOM transformation along the aquatic continuum, most importantly about the role of flocculation and phytoplankton-derived DOM.

\renewcommand{\abstractname}{Acknowledgements}
\begin{abstract}
	This work was supported by the IMAGE (09-067259) from the Danish Council for Strategic Research (S.M.). C.S. was funded by the Danish Research Council for Independent Research (DFF-1323-00336) P. M. was supported by a postdoctoral fellowship from The Natural Sciences and Engineering Research Council of Canada (NSERC). E.A. and C.S. was supported by the BONUS COCOA project (grant agreement 2112932-1), funded jointly by the EU and Danish Research Council. We acknowledge Ciarán Murray for helpful comments on the manuscript and Claudine Ouellet who greatly helped with data collection.
\end{abstract}
