%!TEX root = massicotte_et_al.tex

\section*{Introduction}
\label{sec:Introduction}

Physico-chemical characteristics of dissolved organic matter (DOM) are important properties that drive the functioning of aquatic ecosystems at different levels. The carbon associated with DOM, dissolved organic carbon (DOC), is the main source of metabolic substrates for heterotrophic bacteria and influences the composition of aquatic microbial communities \citep{Findlay2003}. Additionally, the chromophoric fraction of the DOM pool (CDOM) is a major driver of underwater light penetration \citep{Kirk1994} which modulates primary production \citep{Markager2004, Thrane2014, Seekell2015}, photochemistry and protects aquatic organisms against harmful ultraviolet (UV) radiation \citep{Rautio2002}.

In recent decades, climate changes, eutrophication and changes in land use have contributed to increased inputs of colored terrestrial DOM in aquatic ecosystems \citep{Roulet2006, Massicotte2013RSE, Weyhenmeyer2014, Haaland2010}. This has important consequences since the transformation of even a small fraction of the DOM pool can potentially have large impacts on ecosystem functioning \citep{Prairie2008}. Increases in CO2 emissions \citep{Lapierre2013} and reduction in primary production due to light shading \citep{Seekell2015, Thrane2014} have already been documented as consequences of increases in terrestrial DOM at local and regional scales. However, generalizing these effects to global scales is a difficult task because our current understanding on the fate and dynamics of DOM along the complete aquatic continuum gradient (from headwater lakes to oceans) is limited. Since most studies about the fate of DOM are either ecosystem or site specific, there is an indisputable need for integrative studies that will unify existing knowledge to better understand the fate and the dynamic of DOM from a broader perspective during its transport from headwaters lakes to oceans.

DOC and CDOM properties are now routinely measured in many ecological studies. This creates an opportunity to explore the factors regulating the dynamic of the DOM pool at global scales. In this study we performed an extensive literature survey ($n$ = 12808) to extract datasets containing both DOC and CDOM absorption measurements to gain insights about the spatial distribution and the compositional characteristics of the DOM pool during its transition along the complete aquatic continuum. We hypothesized that a strong relationship between DOC and absorption properties of CDOM would be a common characteristic in aquatic ecosystems receiving large amount of colored DOM from their surrounding terrestrial environments. We further expected that the robustness of the observed relationship would weaken as DOM is transformed and degraded during its transition toward oceans because photochemical processes such as photobleaching would remove CDOM preferentially over DOC. An another key objective of this work was to gather and harmonize available published information to help the community to formulate and test new hypotheses about DOM biogeochemical cycling at global scales along the aquatic continuum.
